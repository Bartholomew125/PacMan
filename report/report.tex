%%%%%%%%%%%%%%%%%%%%%%%%%%%%% Define Article %%%%%%%%%%%%%%%%%%%%%%%%%%%%%%%%%%
\documentclass{article}
%%%%%%%%%%%%%%%%%%%%%%%%%%%%%%%%%%%%%%%%%%%%%%%%%%%%%%%%%%%%%%%%%%%%%%%%%%%%%%%

%%%%%%%%%%%%%%%%%%%%%%%%%%%%% Using Packages %%%%%%%%%%%%%%%%%%%%%%%%%%%%%%%%%%
\usepackage{fancyhdr}
\usepackage{lastpage}
\usepackage{titling}
\usepackage[danish]{babel}
\usepackage{geometry}
\usepackage{graphicx}
\usepackage{amssymb}
\usepackage{amsmath}
\usepackage{amsthm}
\usepackage{empheq}
\usepackage{mdframed}
\usepackage{booktabs}
\usepackage{lipsum}
\usepackage{graphicx}
\usepackage{color}
\usepackage{psfrag}
\usepackage{pgfplots}
\usepackage{bm}
\usepackage{hyperref}
\usepackage{minted}
%%%%%%%%%%%%%%%%%%%%%%%%%%%%%%%%%%%%%%%%%%%%%%%%%%%%%%%%%%%%%%%%%%%%%%%%%%%%%%%

% Other Settings

%%%%%%%%%%%%%%%%%%%%%%%%%% Page Setting %%%%%%%%%%%%%%%%%%%%%%%%%%%%%%%%%%%%%%%
\geometry{a4paper}


%%%%%%%%%%%%%%%%%%%%%%%%%% Styles %%%%%%%%%%%%%%%%%%%%%%%%%%%%%%%%%%%%%%%%%%%%%
\usemintedstyle{borland}
%%%%%%%%%%%%%%%%%%%%%%%%%% Define some useful colors %%%%%%%%%%%%%%%%%%%%%%%%%%
\definecolor{ocre}{RGB}{243,102,25}
\definecolor{mygray}{RGB}{243,243,244}
\definecolor{deepGreen}{RGB}{26,111,0}
\definecolor{shallowGreen}{RGB}{235,255,255}
\definecolor{deepBlue}{RGB}{61,124,222}
\definecolor{shallowBlue}{RGB}{235,249,255}
%%%%%%%%%%%%%%%%%%%%%%%%%%%%%%%%%%%%%%%%%%%%%%%%%%%%%%%%%%%%%%%%%%%%%%%%%%%%%%%

%%%%%%%%%%%%%%%%%%%%%%%%%%% Define codecomment command %%%%%%%%%%%%%%%%%%%%%%%%
\newcommand{\code}[1]{\small\mintinline[xleftmargin=2em, xrightmargin=2em, breaklines]{sql}{#1}}
\newcommand{\snippet}[3]{\inputminted[firstline=#1,lastline=#2,linenos, xleftmargin=1.5em, breaklines]{java}{#3}}
%%%%%%%%%%%%%%%%%%%%%%%%%%%%%%%%%%%%%%%%%%%%%%%%%%%%%%%%%%%%%%%%%%%%%%%%%%%%%%%

%%%%%%%%%%%%%%%%%%%%%%%%%% Define an orangebox command %%%%%%%%%%%%%%%%%%%%%%%%
\newcommand\orangebox[1]{\fcolorbox{ocre}{mygray}{\hspace{1em}#1\hspace{1em}}}
%%%%%%%%%%%%%%%%%%%%%%%%%%%%%%%%%%%%%%%%%%%%%%%%%%%%%%%%%%%%%%%%%%%%%%%%%%%%%%%

%%%%%%%%%%%%%%%%%%%%%%%%%%%% English Environments %%%%%%%%%%%%%%%%%%%%%%%%%%%%%
\newtheoremstyle{mytheoremstyle}{3pt}{3pt}{\normalfont}{0cm}{\rmfamily\bfseries}{}{1em}{{\color{black}\thmname{#1}~\thmnumber{#2}}\thmnote{\,--\,#3}}
\newtheoremstyle{myproblemstyle}{3pt}{3pt}{\normalfont}{0cm}{\rmfamily\bfseries}{}{1em}{{\color{black}\thmname{#1}~\thmnumber{#2}}\thmnote{\,--\,#3}}
\theoremstyle{mytheoremstyle}
\newmdtheoremenv[linewidth=1pt,backgroundcolor=shallowGreen,linecolor=deepGreen,leftmargin=0pt,innerleftmargin=20pt,innerrightmargin=20pt,]{theorem}{Theorem}[section]
\theoremstyle{mytheoremstyle}
\newmdtheoremenv[linewidth=1pt,backgroundcolor=shallowBlue,linecolor=deepBlue,leftmargin=0pt,innerleftmargin=20pt,innerrightmargin=20pt,]{definition}{Definition}[section]
\theoremstyle{myproblemstyle}
\newmdtheoremenv[linecolor=black,leftmargin=0pt,innerleftmargin=10pt,innerrightmargin=10pt,]{problem}{Problem}[section]
%%%%%%%%%%%%%%%%%%%%%%%%%%%%%%%%%%%%%%%%%%%%%%%%%%%%%%%%%%%%%%%%%%%%%%%%%%%%%%%

%%%%%%%%%%%%%%%%%%%%%%%%%%%%%%% Plotting Settings %%%%%%%%%%%%%%%%%%%%%%%%%%%%%
\usepgfplotslibrary{colorbrewer}
\pgfplotsset{width=8cm,compat=1.9}
%%%%%%%%%%%%%%%%%%%%%%%%%%%%%%%%%%%%%%%%%%%%%%%%%%%%%%%%%%%%%%%%%%%%%%%%%%%%%%%

%%%%%%%%%%%%%%%%%%%%%%%%%%%%%%% Title & Author %%%%%%%%%%%%%%%%%%%%%%%%%%%%%%%%
\title{\textbf{Objektorienteret Programmering Projekt PacMan}}
\author{Andreas K. L. \quad Aske W. F. \quad Magnus R. K.}
%%%%%%%%%%%%%%%%%%%%%%%%%%%%%%%%%%%%%%%%%%%%%%%%%%%%%%%%%%%%%%%%%%%%%%%%%%%%%%%

\begin{document}
\pagenumbering{gobble}
\begin{titlepage}
    \maketitle
    \begin{figure}[H]
        \begin{center}
            \includegraphics[width=0.8\textwidth]{figures/frontpage-image.png}
        \end{center}
    \end{figure}
\end{titlepage}
    \clearpage
    \newpage
    \pagenumbering{arabic}
    \setcounter{page}{1}

    % Only footer, no header
    \pagestyle{fancy}
    % \fancyhf{} 
    \fancyfoot[C]{\textbf{\thepage}\ of \pageref{LastPage}}
    % \renewcommand{\headrulewidth}{0pt}  

    \tableofcontents
    \newpage
\section{Projektbeskrivelse}\label{sec:Beskrivelse} % (fold)

Til fordel for at sikre en så tro kopi til orignalen som overhovedet muligt, er dette projekts hovedformål, at udvilke spilfunktionalitet mht. kravsspecifikationen. Målet herefter, er at udvide både spillets funktionalitet samt dets brugervenlighed.

Eventuelle afvigelser fra kravsspecifikationen ses dokumenteret/diskuteret i det følgende.
% \begin{itemize}
%   \item Antag, at læseren af jeres rapport har læst kravsspecifikationen fra
%   projektbeskrivelsen i slutningen af dette dokument, og undlad at gentage
%   unødvendige detaljer derfra.
%   \item Formålet med denne sektion er at dokumentere eventuelle afvigelser fra
%   kravsspecifikationen.
%   \item Har I, for eksempel, gjort jer forsimplende antagelser ift.
%   specifikationen? Eller tolket eventuelle tvetydigheder i specifikationen?
%   Tilføjet nye krav? Beskriv hvilke og hvordan I evt. har tolket
%   specifikationen.
%   \item Hold det kort, især hvis I har få afvigelser.
% \end{itemize}
\newpage
\section{design}\label{sec:design} % (fold)
% section Beskrivelse (end)

\begin{itemize}
  \item Inkludér et UML-diagram og en beskrivelse af jeres design.
  \item Giv en kort beskrivelse af jeres diagram:
  \begin{itemize}
    \item Hvad er de forskellige dele?
    \item Har I anvendt designmønstre i jeres design? I så fald, hvor i
    diagrammet findes disse? Det er ikke et krav at anvende designmønstre, men
    kan være en god idé.
  \end{itemize}
  \item Dokumentér designbeslutninger hvor I har anvendt SOLID, DRY, eller andre
  OO-principper. 
  \item Hvis I i løbet af projektet har forfinet jeres design,
  giv da en kort beskrivelse af hvilke ændringer I har foretaget og hvorfor.
\end{itemize}
% section design (end)

\section{Implementation}\label{sec:Implementation} % (fold)
\begin{itemize}
  \item Formålet med denne rapportsektion er at give den interesserede læser et
  overblik over de interessante implementationsdetaljer, som er værd at kigge
  nærmere på i jeres kodebase, samt nødvendige detaljer for at køre jeres kode.
  \item Angiv hvilken version af Java I har brugt til at teste og kompilere
  jeres kode, og inkludérkorte instruktioner til hvordan man kompilerer og kører
  koden.
  \item Giv en beskrivelse på højniveau af interessante implementationsaspekter.
  F.eks., aspekter, I har brugt særligt meget tid eller energi på.
  \item Det kunne f.eks. være mere avancerede aspekter såsom hvordan I håndterer
  AI, hvordan I håndterer spilhandlinger, animation, eller andet.
  \item Hold beskrivelsen overordnet. Vi kan læse jeres kode for detaljerne.
\end{itemize}
% section Implementation (end)

\section{Kvalitetssikring}\label{sec:Kvalitetssikring} % (fold)
\begin{itemize}
  \item Beskriv hvordan I har testet, at jeres kode lever op til
  kravsspecifikationen. Har I, f.eks., benyttet unit tests? Manuelle tests?
  \item Ville I have taget en anden tilgang til kvalitetssikring hvis I skulle
  designe og implementere projektet forfra?
\end{itemize}
% section Kvalitetssikring (end)

\section{Proces}\label{sec:Proces} % (fold)
\begin{itemize}
  \item Arbejdede I i faser i løbet af projektet?
  \item Hvordan gik samarbejdet, og hvordan sikrede I lige deltagelse?
  \item Brugte I tekniske værktøjer til at få samarbejdet til at glide nemmere
  på tværs af maskiner?
  \item Har I brugt AI som støtte under udviklingen af jeres projekt? I så fald,
  hvordan?
\end{itemize}
% section Proces (end)

\section{Diskussion}\label{sec:Diskussion} % (fold)
\begin{itemize}
  \item Ville I gøre noget anderledes hvis I skulle implementere projektet
  forfra?
  \item Var der dele af projektbeskrivelsen I ikke nåede? I så fald, hvordan er
  disse dele kompatible med jeres design? Ville I foretage ændringer for at
  imødekomme ændringer?
\end{itemize}
% section Diskussion (end)

\end{document}
